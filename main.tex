\documentclass[a4paper, 11pt]{article}

\newcommand{\coursename}{Guide de survie \LaTeX}
\newcommand{\professor}{Kevin Grandjean}
\newcommand{\me}{Kevin Grandjean}
\newcommand{\term}{Automne 2014}

\usepackage[utf8]{inputenc}
\usepackage[french]{babel}
\usepackage{notes}

\lstnewenvironment{TeXlstlisting}{\lstset{language=[LaTeX]TeX}}{}

%Headers
\chead{\coursename}
\lhead{\term}

%%%%%TITLE%%%%%
\newcommand{\notefront} {
\pagenumbering{roman}
\begin{center}
\textbf{{\Huge{\noun{\coursename}}}}\\ \vspace{0.1in}
{\noun \professor} \ $\bullet$ \ {\noun \term} \ $\bullet$ \ {\noun {Maison}} \\
\end{center}
}

\title{Template: Survival Guide}

\begin{document}
\notefront
\tocandfigures

\doabstract{Le but de ce document est d'avoir une base pour travailler et comprendre les différentes commandes qui composent le template.}

\section{Définitions mathématiques} % (fold)
\label{sec:d_finitions_math_matiques}
Les définitions mathématiques peuvent être utilisées de la manière suivante:
\begin{lstlisting}[language=TeX]
\begin{defn}[addition]\label{addition}
Une addition se fait avec le signe ``+''.
\end{defn}
\end{lstlisting}
Résultat:
\begin{defn}[addition]\label{addition}
Une addition se fait avec le signe ``+''.
\end{defn}
% section d_finitions_math_matiques (end)

\section{Alignement pour développement mathématique} % (fold)
\label{sec:alignement_pour_d_veloppement_math_matique}
On peut aligner des équations pour que leur signe \og = \fg soit verticalement aligné.
Pour définir l'emplacement qui sera aligné, il faut utiliser le symbole \&.
\begin{TeXlstlisting}
\begin{align*}
3 & = 1 + 2 \\
&= 1 + 1 + 1
\end{align*}
\end{TeXlstlisting}

Résultat:
\begin{align*}
3 & = 1 + 2 \\
&= 1 + 1 + 1
\end{align*}
% section alignement_pour_d_veloppement_math_matique (end)

\section{Références} % (fold)
\label{sec:r_f_rences}
On peut utiliser la commande \\label\{name\} pour indiquer une ancre à cet endroit. On peut ensuite pointer cet emplacement avec la commande \\nameref\{name\} qui permettra de directement se déplacer vers l'endroit pointé.
% section r_f_rences (end)

\section{Affichage de code} % (fold)
\label{sec:affichage_de_code}
Pour afficher du code, il suffit d'utiliser la commande suivante:

\begin{TeXlstlisting}
\begin{lstlisting}[language=java]
void main(String[] args){
	System.out.println("Hello");
}
\end{lstlisting}
\end{TeXlstlisting}

Résultat:
\begin{lstlisting}[language=java]
void main(String[] args){
	System.out.println("Hello");
}
\end{lstlisting}
% section affichage_de_code (end)

\section{Liste des objets mathématiques} % (fold)
\label{sec:liste_des_objets_math_matiques}
\subsection{Bracket en bas}
\begin{verbatim}
\ub{h}
\end{verbatim}
$$\ub{h}$$

\subsection{Bracket à gauche}
\begin{verbatim}
\piecewise{H}{e}{l}{l}
\end{verbatim}
$$\piecewise{H}{e}{l}{l}$$

\subsection{Vecteurs (valeurs 2)}
\begin{verbatim}
\comb{4}{1}
\end{verbatim}
$$\comb{4}{1}$$

\subsection{Parenthèse anguleuse}
\begin{verbatim}
\sip{2}{3}
\end{verbatim}
$$\sip{2}{3}$$

\subsection{Fraction}
\begin{verbatim}
\f{H}{X}
\end{verbatim}
$$\f{H}{X}$$

\subsection{Ensembles de nombres}
\begin{verbatim}
\Z \Q \R \C \N
\end{verbatim}
$$\Z \Q \R \C \N$$

\subsection{Epsilon}
\begin{verbatim}
\e
\end{verbatim}
$$\e$$

\subsection{Union et Inter}
\begin{verbatim}
\union \inter
\end{verbatim}
$$\union \inter$$

\subsection{Into}
\begin{verbatim}
\into
\end{verbatim}
$$\into$$

\subsection{Set de nombres}
\begin{verbatim}
\nset{a}
\setk{a}{b}
\bnset{a}
\bset{a}{b}
\end{verbatim}
$$\nset{a}$$
$$\setk{a}{b}$$
$$\bnset{a}$$
$$\bset{a}{b}$$

\subsection{1 to n}
\begin{verbatim}
\ton{H}
\end{verbatim}
$$\ton{H}$$

\subsection{1 to k}
\begin{verbatim}
\tok{H}{k}
\end{verbatim}
$$\tok{H}{k}$$

\subsection{Opérateurs textuels (disc, span, rank, proj et perp)}
\begin{verbatim}
\disc \spn \rank \proj \prp
\end{verbatim}
$$\disc \spn \rank \proj \prp$$

\subsection{Underline}
\begin{verbatim}
\ux \ua \uu
\end{verbatim}
$$\ux \ua \uu$$

\subsection{Dérivées (Leibniz)}
\begin{verbatim}
\pfx \pfy \px \pxn{H} \py \jacu \jacx
\end{verbatim}
$$\pfx \pfy \px \pxn{H} \py \jacu \jacx$$

\subsection{Vecteurs}
\begin{verbatim}
\vzero \va \vb \vc \vd \ve \vh \vn \vs \vu \vv \vw \vx \vy \vz
\end{verbatim}
$$\vzero \va \vb \vc \vd \ve \vh \vn \vs \vu \vv \vw \vx \vy \vz$$
% section liste_des_objets_math_matiques (end)

\section{Utilisation de styles} % (fold)
Pour utiliser les styles, il faut utiliser le code suivant en remplaçant le style par celui voulu:
\begin{TeXlstlisting}
\begin{style}
Texte
\end{style}
\end{TeXlstlisting}
\label{sec:utilisation_de_styles}
\subsection{Corollaire}
\label{sub:corollaire}
Remplacer par cor:
\begin{cor}
Ceci est un test.
\end{cor}

\subsection{Note}
\label{sub:note}
Remplacer par note:
\begin{note}
Ceci est un test.
\end{note}

\subsection{Notation} % (fold)
\label{sub:notation}
Remplacer par notation:
\begin{notation}
Ceci est un test.
\end{notation}
% subsection notation (end)

\subsection{Exercice} % (fold)
\label{sub:exercice}
Remplacer par exercise(terme anglais):
\begin{exercise}
Ceci est un test.
\end{exercise}
% subsection exercice (end)

\subsection{Solution} % (fold)
\label{sub:solution}
Remplacer par solution:
\begin{solution}
Ceci est un test.
\end{solution}
% subsection solution (end)

\subsection{Abstrait} % (fold)
\label{sub:abstrait}
Utiliser \textbackslash doabstract\{Texte\} pour rendre le premier texte abstrait.
%Pour les besoins du document, doabstract n'a pas été utilisé.
\begin{abstract}
Ceci est un test.
\end{abstract}
% subsection abstrait (end)
% section utilisation_de_styles (end)

\end{document}